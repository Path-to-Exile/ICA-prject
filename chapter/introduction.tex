% !TEX root = ‎⁨‎⁨/../../master.tex

\section{Introduction}

Consider the following scenario: You are an avid Magic The Gathering (MTG) player. One of the things you love the most about the game is the art on the cards. You have therefore downloaded the art of 5 of your favorite MTG cards. To your horror you realize one day that all the pictures are mixed as in Figure \ref{fig:toy-example}. Unfortunately you cannot remember where you obtained the pictures in the first place so the question thus becomes: \textit{Can the pictures be unmixed}?

\begin{figure}
	\centering
		\includegraphics[width=0.95\columnwidth]{figures/comparison.pdf}
	\caption{The pictures are not in order anymore and two of the pictures are the negative of the original. This is be elaborated on in Section \ref{sec:ambiguities-ica}. All rights reserved \href{https://magic.wizards.com/en}{Wizards of the Coast}.}
	\label{fig:toy-example}
\end{figure}

The answer turns out to be yes. A picture can be represented by a sequence of numbers, one for each pixel, signifying the color of the given pixel. This way each of the pictures in the second row of Figure \ref{fig:toy-example}, denoted by $x_i$, $i=1,\ldots, 5$, can be considered as a linear mix of the five original pictures denoted by $s_i$. With this we have for each pixel $t$:
	\begin{align*}
		x_1(t)&=a_{11}s_1(t)+a_{12}s_2(t)+\ldots+a_{15}s_5(t)\\
		x_2(t)&=a_{21}s_1(t)+a_{22}s_2(t)+\ldots+a_{25}s_5(t)\\
		&\vdots \\
		x_5(t)&=a_{51}s_1(t)+a_{52}s_2(t)+\ldots+a_{55}s_5(t).
	\end{align*} 
where $a_{ij}$ $i,j=1,\ldots, 5$ are some real-valued parameters the computer mixed each picture with. If we knew the $a_{ij}$ parameters our job would be done as our problem would be equivalent to solving five linear equations with five unknowns, but the snag is precisely that the \textit{parameters are unknown}. 

It turns that out under some rather mild assumptions it is possible to recover the original sources $\widetilde{s}_i$ with \textit{Independent Component Analysis}, henceforth ICA for short. The result of ICA can be seen in the third row of pictures in Figure \ref{fig:toy-example}.

ICA has over the last 40 years established itself as a mature field of research with a wide array of applications, with more direct usefulness than unmixing MTG card art, from econometrics to recordings of the brain \cite[pp.11-12]{hyvarinen2000}. More precisely we will be analyzing electroencephalographic (EEG) data. EEG data consists of recordings of electrical potentials in different locations on the scalp. Supposedly these potentials are generated by a mixture of underlying brain and muscle activity \cite[p.150]{hyvarinen2000}. The problem is similar to the one above: We would like to know the original sources, but the mixing of the sources is unknown.

\subsection{Outline of Objective}

We will in this paper test different ICA methods by comparing the performance achieved in different pipeline setups. In this paper performance is measured by a classifiers ability to accurately guess certain motor imagery actions using features extracted from the individual unmixed signals from the ICA's.

When assessing accuracy in experimental setups like these, there are many choices to be made in regards to parameters and hyper-parameters. It is clear that different choices can lead to different conclusions. The question of scientific objectivity is not new and it is hardly positioned to solved anytime soon\footnote{see e.g. \cite{sep-scientific-realism} for an overview of the debate.}. 

But no matter if you are standing in the Popperian or the Kuhnian corner, it is important to be aware of how otherwise sensible choices can color our perception of different models and their mutual strength. This paper is therefore in part a small scale attempt to do just that. By exploring different parameter settings we will examine whether or not the relative performance of different ICA's is consistent across different pipelines. In the end we will hopefully be able to identify which ICA performs the best on our EEG data. In summary our two overarching goals are:

\begin{enumerate}[label=\alph*),noitemsep]
	\item Examine if different choices in hyper-parameters result in different relative score.
	\item Assess the performance of different ICA methods on EEG data.
\end{enumerate}